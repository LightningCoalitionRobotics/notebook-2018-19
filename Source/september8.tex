\documentclass[12pt]{article}
\begin{document}

September 8, 2018 - Griffin Dugan

\section{Our Plan:} 
\begin{itemize}
	\item Take notes on the new rules. 
	\item Brainstorm! 
	\item Begin strategizing.
	\item Decide what parts need to be focused on. 
\end{itemize}

As it is the kickoff, we need to watch the challenge video and take it apart. Then we need to begin strategizing and brainstorming. 

\section{What We Got Done:}
\begin{itemize}
	\item Notes were taken on the rules.
	\item Brainstormed! 
	\item Started to strategize. 
	\item Decided what parts needed to be focused on. 
\end{itemize}

When we were taking notes on the rules, some of the rules surprised us, like how there is now a weight limit. This will be interesting as our robot was very heavy last year. Then, we decided to think about what to focus on for the competition. The latching and detaching to the hooks seem to be very important as one can get 75 points by doing both. From this, we decided that we need to figure out how to connect our robot to the hook. We thought that we should raise up a hook and attach to the hook in the center, then we would raise up the robot. 

\section{What We Didn't Get Done:} 
\begin{itemize}
	\item Nothing! 
\end{itemize}

As it was only the kickoff, we weren't able to have much to do except brainstorm. 

\section{Next Practice:}
\begin{itemize}
	\item Fully take apart the challenge. 
	\item Start planning. 
	\item Decide how we are going to work this year. 
\end{itemize}

The next practice is currently unknown as we haven't had a full meeting yet. Even so, we will want to fully take apart the challenge and gather what things give us points. We will want to start planning. After that, we will want to decide how we will work this year buy deciding what things need to be done and what will take the most time.

\end{document}