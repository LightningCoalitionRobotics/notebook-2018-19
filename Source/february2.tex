%---------------------------------------------------------------------------------------------------------------------------------------------------------------------------------------------------------------------------------------------------------------------------------------------------------------------------------------------------------------------------------------------------
% Welcome to the notebook template. When every you see this symbol, %, it means that there is a comment and that it will not show up on the finished page. The notebook as shown is written in the coding language, LaTex; this language is an official notebook language for engineers, which is the main reason why we are using it. You can make a copy of this document if you like, but be sure to change everything that the comments tell you to change. This guide is made so that you (the person writing the notebook) doesn't have to know LaTex. But to make the notebook into the pdf that it needs to be, please download Texmaker (http://www.xm1math.net/texmaker/index.html). It allows one to code in LaTex and transfer a LaTex document to PDF form. Please also post both forms of your notebook in the NOTEBOOK folder on Google Drive. Remember to name it by date (e.g. september14.pdf). You can delete this comment section if you want. (everything in between the big long lines.) Good luck. Have fun.
% - Griffin Dugan.
%---------------------------------------------------------------------------------------------------------------------------------------------------------------------------------------------------------------------------------------------------------------------------------------------------------------------------------------------------------------------------------------------------


\documentclass[12pt]{article}
\usepackage{caption}
\usepackage{float}
\usepackage{graphicx}
\usepackage{fancyhdr}
\pagestyle{fancy}
\pagenumbering{Roman}
\renewcommand{\headrulewidth}{1pt}
\renewcommand{\footrulewidth}{1pt}
\setlength{\headheight}{25pt}
\rhead{\textbf{Lightning Coalition Robotics}}
\cfoot{}
\rfoot{\thepage}
\begin{document}

% Add date (e.g. September 14, 2018) and then your name/all authors.
February 2, 2019 - Noah Simon

\section{Our Plan:} % Pretty self explainatory... In this section explain the team's plan.
\begin{itemize}
% After \item, add what you want for the bullet point. (\item adds a new bullet point when you run out.)
	\item Finish Engineering Notebook Google Form
	\item Design and build new lifter
	\item Work on scissor lift
	\item Test API
\end{itemize}

% Now add a paragraph explaining your plan. You should reference the bullet points above.
With two weeks left until our next qualifier, there's a lot still to do. Chris is still working on a Python program to streamline the process of writing notebook entries, Noah is ready to test the API he wrote, and the build team needs to build a new lifter (for the lander hook) and finish the scissor lift.

\section{What We Got Done:} % Just like the above, explain what the team got done during practice.
\begin{itemize}
% After \item, add what you want for the bullet point. (\item adds a new bullet point when you run out.)
	\item Thundercats!
	\item Reached out about reaching out
	\item Progress on scissor lift
	\item Designed poster
	\item Prepared scouters
	\item Progress on Google Form
\end{itemize}

% Now add a paragraph explaining what you got done and the reasons behind it. You should reference the bullet points above.
Today, we spoke with the FLL robotics coach at our school (not the team that visited today) about an outreach opportunity for March. The builders have almost the scissor lift, and Chris has almost finished his notebook program. The administrative team talked about our presentation for the qualifier and about improving the way we scout.

\subsection{The Thundercats' Visit}
We had a lot of fun with FLL Team \#31745, the Thundercats! We presented to them our basic workflow as an FTC team, and compared it to what they already do. We showed them the Engineering Notebook and how that works, and we let them drive around the robot, much to their delight. Please see the Outreach page for more details.

\section{What We Didn't Get Done:} % Explain what the team didn't get done during practice.
\begin{itemize}
% After \item, add what you want for the bullet point. (\item adds a new bullet point when you run out.)
	\item Finish Google Form
	\item Test API
\end{itemize}

% Now add a paragraph explaining what you didn't get done and the reasons behind it. You should reference the bullet points above.
While we had a lot of fun hanging out with the Thundercats, the visit prevented us from getting some of the work done. Noah couldn't test his code because he was busy presenting and the robot was being worked on, and the Google Form wasn't completed for a similar reason.

\section{Next Practice:}
\begin{itemize}
% After \item, add what you want for the bullet point. (\item adds a new bullet point when you run out.)
	\item Buy domain for website
	\item Test API
	\item Put robot back together
\end{itemize}

% Now add a paragraph explaining what the plan for next practice should be. You should reference the bullet points above.
The next practice is next Friday, February 8. % And then continue the paragraph.
Fridays are our short days, so we shouldn't set our hopes too high, but we would at least like to finally test the Android API and buy the domain for our website. Also, all the parts need to be back on the robot so it's ready for the following weekend.

\end{document}
