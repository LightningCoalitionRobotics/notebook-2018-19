%---------------------------------------------------------------------------------------------------------------------------------------------------------------------------------------------------------------------------------------------------------------------------------------------------------------------------------------------------------------------------------------------------
% Welcome to the notebook template. When every you see this symbol, %, it means that there is a comment and that it will not show up on the finished page. The notebook as shown is written in the coding language, LaTex; this language is an official notebook language for engineers, which is the main reason why we are using it. You can make a copy of this document if you like, but be sure to change everything that the comments tell you to change. This guide is made so that you (the person writing the notebook) doesn't have to know LaTex. But to make the notebook into the pdf that it needs to be, please download "Texmaker" (http://www.xm1math.net/texmaker/index.html). It allows one to code in LaTex and transfer a LaTex document to PDF form. Or you could install "TeX, LaTex Viewer and Editor" (https://chrome.google.com/webstore/detail/tex-latex-viewer-and-edit/lmbknmfadpeadepgoblginkbiljjpcea?hl=en-US). Please also post both forms of your notebook in the NOTEBOOK folder on Google Drive. Remember to name it by date (e.g. september14.pdf). You can delete this comment section if you want. (everything in between the big long lines.) Good luck. Have fun.
% - Griffin Dugan.
%---------------------------------------------------------------------------------------------------------------------------------------------------------------------------------------------------------------------------------------------------------------------------------------------------------------------------------------------------------------------------------------------------


\documentclass[12pt]{article}
\usepackage{caption}
\usepackage{float}
\usepackage{graphicx}
\usepackage{fancyhdr}
\pagestyle{fancy}
\pagenumbering{Roman}
\renewcommand{\headrulewidth}{1pt}
\renewcommand{\footrulewidth}{1pt}
\setlength{\headheight}{25pt}
\rhead{\textbf{Lightning Coalition Robotics}}
\cfoot{}
\rfoot{\thepage}
\begin{document}

% Add date (e.g. September 14, 2018) and then your name/all authors.
October 5, 2018 - Noah Simon

\section{Our Plan:} % Pretty self explainatory... In this section explain the team's plan.
\begin{itemize}
% After \item, add what you want for the bullet point. (\item adds a new bullet point when you run out.)
	\item Unpack Stuff
	\item Take inventory
	\item Get coding infrastructure set up on members' computers
\end{itemize}

% Now add a paragraph explaining your plan. You should reference the bullet points above.
We just got our shipments in, including both board- and robot-building materials. We need to take all that out, and make sure we have everything that we're supposed to have. Also, we need to get Android Studio set up on all the coders' computers. 😫😫

\section{What We Got Done:} % Just like the above, explain what the team got done during practice.
\begin{itemize}
% After \item, add what you want for the bullet point. (\item adds a new bullet point when you run out.)
	\item Unpack Stuff
	\item Take inventory
	\item Ideas for claw mechanism
\end{itemize}

% Now add a paragraph explaining what you got done and the reasons behind it. You should reference the bullet points above.
We basically dumped everything out on the floor, and figured out what we have. It turns out we are missing all of the lander materials due to a backorder. Nothing we can do about that except wait. We also messed around with different ways to construct a claw mechanism for picking up scoring elements.

\section{What We Didn't Get Done:} % Explain what the team didn't get done during practice.
\begin{itemize}
% After \item, add what you want for the bullet point. (\item adds a new bullet point when you run out.)
	\item Get coding infrastructure set up on member's computers.
\end{itemize}

% Now add a paragraph explaining what you didn't get done and the reasons behind it. You should reference the bullet points above.
This was a short practice, only 45 minutes, so we weren't expecting to get too much done with regards to actually building things. However, how hard can setting up Android Studio be? It turns out: very difficult. Android Studio is already complicated to set up, and this was exacerbated by our school's incredibly slow WiFi connection. 

\section{Next Practice:}
\begin{itemize}
% After \item, add what you want for the bullet point. (\item adds a new bullet point when you run out.)
	\item Finish setting up coding infrastructure
	\item Start building a chassis
	\item Inventory motors, servos, etc.
	\item Order any of the above items we need
\end{itemize}

% Now add a paragraph explaining what the plan for next practice should be. You should reference the bullet points above.
The next practice is tomorrow, October 6. % And then continue the paragraph.
Four hours tomorrow should hopefully be enough time for our lead programmer to get the development environment set up for everyone. The builders can finally start actually building since we have a better idea of what we are building for. We also should take a look at how many hardware components we have and see if we need to order more.

\end{document}