
\documentclass[12pt]{article}
\usepackage{caption}
\usepackage{float}
\usepackage{graphicx}
\usepackage{fancyhdr}
\pagestyle{fancy}
\pagenumbering{Roman}
\renewcommand{\headrulewidth}{1pt}
\renewcommand{\footrulewidth}{1pt}
\setlength{\headheight}{25pt}
\rhead{\textbf{Lightning Coalition Robotics}}
\cfoot{}
\rfoot{\thepage}

\begin{document}

% Add date (e.g. September 14, 2018) and then your name/all authors.
October 25 2018 - William Nolan

\section{Our Plan:} % Pretty self explainatory... In this section explain the team's plan.
\begin{itemize}
% After \item, add what you want for the bullet point. (\item adds a new bullet point when you run out.)
    \item In the beginning, two groups were formed.  The first group focused on assembling the robot testing enviroment, and the second began work on the robot's claw.
    \item Group 1 began with inventorying all of the part of the robot's enviroment.  They put together the central box frame, but couldn't complete it due to the fact that we didn't have the right tools to screw in the nylon hex nuts.
    \item Group 2 decided to break into individuals, and each design their own ideal prototype robot claw. Afterwards they would debate the merits of their designs.
    \item Gabriel Ruoff worked independently to build the robot's chasis and a side tank tread wheel. This was completed.
\end{itemize}

% Now add a paragraph explaining your plan. You should reference the bullet points above.
After about 20 minutes, Group 2 was to come back together and discuss the final design of the robot.

\section{What We Got Done:} % Just like the above, explain what the team got done during practice.
\begin{itemize}
% After \item, add what you want for the bullet point. (\item adds a new bullet point when you run out.)
    \item Group 1 inventoryed all of the materials delivered that were not going to be part of the robot. They completed the intial box for the lander in the Robo Ruckus Course.
    \item Group 2's individual designed their own prototype claw and attachments to the robot and began debated their designs.  For example, Griffin designed a type of ground device which would used high-speed wheels to pick up items.
\end{itemize}

% Now add a paragraph explaining what you got done and the reasons behind it. You should reference the bullet points above.

\section{What We Didn't Get Done:} % Explain what the team didn't get done during practice.
\begin{itemize}
% After \item, add what you want for the bullet point. (\item adds a new bullet point when you run out.)
    \item As I said before, in the second bullet point, Group 1 was unable to finish the robot enviroment.  Hopefully, a charged dremel and a more precise drill bits will allow for greater progress. 
    \item There was no actual resolution with Group 2.  They each designed a device to obtain items from the ground, but they did not have enough time to determine the best one to implement in our robot.
\end{itemize}

% Now add a paragraph explaining what you didn't get done and the reasons behind it. You should reference the bullet points above.
There wasn't enough time for there to be enough discussion to fully decide on the type of claw to implement on the robot.

\section{Next Practice:}
\begin{itemize}
% After \item, add what you want for the bullet point. (\item adds a new bullet point when you run out.)
    \item All in all, while the strategy of how the course is to be handed was not talked about, the time was very productive and the robot's prototype is nearing completion.  
\end{itemize}

% Now add a paragraph explaining what the plan for next practice should be. You should reference the bullet points above.
The next practice is October 27, 2018.  Then, the team plans to finalize the design of the robot's attachments.

\end{document}

